\documentclass{beamer}
\mode<presentation>
\title{Experiments as functions of state}
\date{2015-01-08}
\begin{document}

\begin{frame}
\titlepage
\end{frame}

\begin{frame}
\frametitle{Warning!}
Some of the terminology used could be wrong.

I'm not a mathematician or a type theorist.
\end{frame}

\section{Motivation}
\begin{frame}
\frametitle{Motivation}
If there was a language for describing markets we could develop algorithms
which operation on statements in that language.  Those algorithms would work
for an arbitrary market.
\end{frame}

\section{Language}
\begin{frame}
\frametitle{Simplification}
Only markets whose selections are \emph{collectively exhaustive} and
\emph{mutually exclusive} are modeled.  Such markets could also be called
\alert{sample spaces} and their selections \alert{outcomes} or
\alert{elementary events}.

\begin{columns}
\column{0.5\textwidth}
Definable:
\begin{itemize}
\item Match result
\item Correct score
\end{itemize}
\column{0.5\textwidth}
Undefinable:
\begin{itemize}
\item Double chance
\item Correct score, can't lose
\end{itemize}
\end{columns}

Markets that cannot be defined in this model can be represented as a
composition of sample spaces.
\end{frame}

\begin{frame}
\frametitle{Composition}
Each market contains selections which are composed of one or more outcomes
from a sample space.

\begin{example}
Correct score, can't lose has a selection for each ``true'' outcome of the
set of ``will the score be $x$--$y$ in the match?'' sample spaces.
\end{example}

\begin{example}
Double chance has a selection that negates a corresponding selection in the
``which team will score the most goals?'' sample space.
\end{example}
\end{frame}

\begin{frame}
\frametitle{Example experiments}
\begin{description}
\item[total goals] len(goals)
\item[correct score] map(partition(goals, 'participant'), len)
\item[most goals] max(map(partition(goals, 'participant'), len))
\end{description}

\begin{description}
\item[len(coll)] the number of elements in \emph{coll}
\item[partition(coll, attr)] a mapping of the values of \emph{attr} to the
collection of elements with those values
\item[map(mapping, fn)] applies \emph{fn} to each value in \emph{mapping}
\item[max(mapping)] the set of keys with the maximum value
\end{description}
\end{frame}

\begin{frame}
\frametitle{Example experiments}
\begin{example}[half-time full-time]
map(partition(filter(goals, period(0)), 'participant'), len) $\times$

map(partition(filter(goals, period(1)), 'participant'), len)
\end{example}

\begin{description}
\item[filter(coll, pred)] elements in \emph{coll} that satisfy \emph{pred}
\item[a $\times$ b] product of \emph{a} and \emph{b}
\end{description}
\end{frame}

\begin{frame}
\frametitle{Functions of state}
The previous examples are unary functions taking the state of an event.  The
state parameter can be omitted because it is always present.

\begin{description}
\item[total goals] len(goals)
\item[total-goals(state)] len(state.goals)
\end{description}
\end{frame}

\begin{frame}
\frametitle{Parameters}
Definitions can contain references to undefined names which are taken to be
parameters to the experiment.

\begin{description}
\item[nth goal scorer] attr(nth(goals, \emph{n}), 'participant')
\item[nth-goal-scorer(state)] \ldots
\end{description}

\begin{description}
\item[nth(coll, n)] the \emph{n}th element of \emph{coll}
\item[attr(elem, attr)] \emph{elem}.\emph{attr}
\end{description}
\end{frame}

\begin{frame}
\frametitle{Parameters type inference}
\begin{description}
\item[nth goal scorer] attr(nth(goals, \emph{n}), 'participant')
\item[first event scorer] attr(nth(\emph{c}, 0), 'participant')
\end{description}

nth takes a collection and a natural number, thus the trader is prompted to
provide a value of the correct type when adding the sample space to a market.
If he does not provide a value a market is created for each possible value of
each parameter.

The algorithms never receive functions with unfixed parameters except
\emph{state}.
\end{frame}

\section{Algorithms}
\begin{frame}
\frametitle{Manipulating functions}
Functions are simply abstract syntax trees.

Most algorithms will simply dispatch on the type of the node and invoke
themselves recursively on children.

These implementations are suitable for memoization.
\end{frame}

\begin{frame}
\frametitle{Human-readable name}
\begin{description}
\item[len(coll)] "total " + describe(coll)
\item[goals] "goals"
\item[partition(coll, attr)] describe(coll) + " by " + attr
\item[map(coll, fn)] describe(fn(coll))
\end{description}

\begin{example}[describe(correct score)]
\begin{enumerate}
\item describe(map(partition(goals, 'participant'), len))
\item describe(len(partition(goals, 'participant')))
\item "total " + describe(partition(goals, 'participant'))
\item "total " + desc(goals) + " by " + "participant"
\item "total goals by participant"
\end{enumerate}
\end{example}
\end{frame}

\end{document}
